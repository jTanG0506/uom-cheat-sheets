\documentclass[a4paper]{article}

\usepackage{mathtools}
\DeclarePairedDelimiter\ceil{\lceil}{\rceil}
\DeclarePairedDelimiter\floor{\lfloor}{\rfloor}

\usepackage[a4paper, margin=0.35in]{geometry}
\usepackage{multicol}
\usepackage{amssymb}
\usepackage{amsmath}
\usepackage{centernot}
\usepackage{framed}
\usepackage{bm}
\usepackage{listings}

\begin{document}

\pagenumbering{gobble}
	
\begin{center}
	\huge{\textbf{algorithms and imperative programming (i)}}\\
	\small{Available at \textsc{jtang.dev/resources}}\\
\end{center}
\begin{multicols}{2}

\begin{framed}
	\begin{center}
		\textbf{\textsc{complexity measures}}
	\end{center}
	\textbf{O$(f)$} denotes a set of functions:\\
	$\{g:\mathbb{N} \rightarrow \mathbb{N} \vert \exists n_0 \in \mathbb{N}, c \in \mathbb{R}^+, \forall n > n_0, g(n) \leq c \cdot f(n) \}$
	\textbf{$\bm{\Omega(f)}$} denotes a set of functions:\\
	$\{g:\mathbb{N} \rightarrow \mathbb{N} \vert \exists n_0 \in \mathbb{N}, c \in \mathbb{R}^+, \forall n > n_0, g(n) \geq c \cdot f(n) \}$
	\textbf{$\bm{\Theta(f)}$} denotes $\bm{O(f) \, \cap \, \Omega(f)}$
\end{framed}

\begin{framed}
	\begin{center}
		\textbf{\textsc{euclid's algorithm}}
	\end{center}
\begin{lstlisting}
Algorithm EuclidGCD(a, b):
  Input: Non-negative integers a and b
  Output: gcd(a, b)
		
  if b = 0 then
    return a
  return EuclidGCD(b, a mod b)
end
\end{lstlisting}

\noindent
\textbf{correctness}\\
Let $d = gcd(a, b)$ and $c = gcd(b, a - rb)$. \\
We need to show that $gcd(a, b) = gcd(b, a - rb)$, so $d = c$.\\
By definition of $d$, we have the number $\frac{(a - rb)}{d} = \frac{a}{d} - r\frac{b}{d}$ is an integer as $d \vert a$ and $d \vert b$ and we have also shown $d \vert a - rb$ hence $d \leq c$.\\
Now by definition of $c$, $\frac{a - rb}{c} = \frac{a}{c} - r\frac{b}{c}$ shows that $c \vert a$ as we know $r\frac{b}{c}$ is an integer and $\frac{a - rb}{c}$ is an integer, so we have $c \leq d$.\\

\noindent
\textbf{complexity}\\
After the first call, the first argument is always larger than the second one. Denote $a_i$ as the first argument of the $i$th recursive call of EuclidGCD. It is clear that the second argument of a recursive call is equal to $a_{i + 1}$ and we also have $$a_{i+2} = a_i \, \text{mod} \, a_{i+1}$$ which implies the sequence $a_i$ is strictly decreasing. We claim that $$a_{i+2} < \frac{1}{2}a_i$$

\noindent
\textbf{case 1}: $a_{i+1} \leq \frac{1}{2}a_i$, since the sequence of $a_i$'s is strictly decreasing, we have $$a_{i+2} < a_{i+1} \leq \frac{1}{2} a_i$$

\noindent
\textbf{case 2}: $a_{i+1} > \frac{1}{2}a_i$, in this case $a_{i + 2} = a_i \, \text{mod} \, a_{i+1}$, so we have $$a_{i+2} = a_i \, \text{mod} \, a_{i+1} = a_i - a_{i + 1} < \frac{1}{2}a_i$$

\noindent
Thus the size of the first argument to the EuclidGCD method decreases by half with every other recursive call. Hence we have $\bm{O(log \max(a, b))}$.
\end{framed}

\begin{framed}
	\begin{center}
		\textbf{\textsc{modular arithmetic}}
	\end{center}
\begin{lstlisting}
Algorithm pow1(a, b, k):
  Input: Integers a, b, k
  Output: a^b mod k
		
  s = 1
  for i from 1 to b
    s = s * a mod k
  return s
end
\end{lstlisting}
\noindent
The number of operations performed here is clearly $O(b)$, therefore the time complexity is $O(2^n)$ as the size of $b$ is $\log_2 b$.\\
\begin{lstlisting}
Algorithm pow2(a, b, k):
  Input: Integers a, b, k
  Output: a^b mod k
		
  d = a, e = b, s = 1
  until e = 0
    if e is odd
      s = s * d mod k
    d = d * d mod k
    e = floor(e / 2)
  return s
end
\end{lstlisting}
\noindent
The number of operations performed here is proportional to the number of times $e$ ($=b$) can be halved before reaching 0, i.e. at most $\ceil{log_2b}$. It follows that this algorithm has running time in $O(n)$.\\

\noindent
\textbf{primitive roots}\\
We say that $g$ is a \textbf{primitive root} with respect to $p$ means that $\mathbb{Z}_p = \{1, 2, \cdots, p - 1\}= \langle g \rangle = \{g^i \, \text{mod} \, p \, \vert \, i \in \mathbb{Z}\}$\\

\begin{lstlisting}
Algorithm dl(y, g, p):
  Input: Integers y, g, p
  Output: x such that y = g^x mod p
	
  a = y mod p	
  for x from 1 to p - 1
    b = pow2(g, x, p)
    if a = b
      return x
  end
end
\end{lstlisting}

\noindent
The number of loop iterations is $O(p)$ and in each iteration, the pow2 call is $O(x)$. So the total number of operations is bounded by $O(px)$ but $x < p$ so this is also bounded by $O(p^2)$ which is $O(4^n)$ as the size of $p$ is $log_2p$.\\

\noindent
\textbf{El Gamal} with private key $x$\\ 
public key $(p, g, y)$ with $y = g^x \mod p$\\
cipher $(a, b)$ with $a = g^k \mod p$ and $b = My^k \mod p$\\
message $M$ = $b/(a^x) \mod p = b(a^x)^{-1} \mod p$
\end{framed}
\end{multicols}
\end{document}