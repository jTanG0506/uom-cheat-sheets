\documentclass[a4paper]{article}

\usepackage{mathtools}
\DeclarePairedDelimiter\ceil{\lceil}{\rceil}
\DeclarePairedDelimiter\floor{\lfloor}{\rfloor}

\usepackage[a4paper, margin=0.35in]{geometry}
\usepackage{multicol}
\usepackage{amssymb}
\usepackage{amsmath}
\usepackage{centernot}
\usepackage{framed}
\usepackage{bm}
\usepackage{listings}

\begin{document}

\pagenumbering{gobble}

\begin{multicols}{2}

\begin{framed}
	\begin{center}
		\textbf{\textsc{determinants and permanents}}
	\end{center}
	\textbf{permutations}\\
	A \textbf{permutation} is a 1-1 map of a set X onto itself. The number of permutations on an $n$-element set is $n!$. A simple inductive proof shows that, for all $n \geq 4$, $2^n \leq n! \leq 2^{n^2}$. That is: $n \mapsto n!$ is $\Omega(2^n)$ and $O(2^{n^2})$.\\
	
	\noindent
	\textbf{transpositions}\\
	A \textbf{transposition} is a permutation of two elements, i.e $\sigma = (\alpha\beta)$.
	
	\noindent
	\textbf{parity}\\
	The \textbf{parity} of a permutation $\sigma$, denoted $sgn(\sigma)$ is 1 if $\sigma$ is the product of an even number of transpositions, -1 otherwise.\\
	
	\noindent
	\textbf{proof that $sgn(\sigma) = \pm 1$}\\
	Let $\sigma \in S_n$. Write $\sigma = (\alpha_1 \alpha_2 \cdots \alpha_m)$. In the view of
	$$(\beta_1 \beta_2 \beta_3 \cdots \beta_k) = (\beta_1 \beta_2)(\beta_1 \beta_3) \cdots (\beta_1 \beta_k)$$
	any permutation of length $k$ can be written as a composite of $k - 1$ transpositions. Now consider when $k$ is even or odd, then the result follows.\\
	
	\noindent
	Note that $sgn(\sigma \cdot \tau) = sgn(\sigma) \cdot sgn(\tau)$ and $sgn(t) = -1$, where $t$ is a transposition.\\
	
	\noindent
	\textbf{matrices}\\
	$$A = 
 	\begin{pmatrix}
 	 a_{1,1} & a_{1,2} & \cdots & a_{1,n} \\
 	 a_{2,1} & a_{2,2} & \cdots & a_{2,n} \\
 	 \vdots  & \vdots  & \ddots & \vdots  \\
 	 a_{n,1} & a_{n,2} & \cdots & a_{n,n} 
 	\end{pmatrix}$$
 	
 	$$det(A) = \sum_{\sigma \in \text{perm}\{1, \cdots, n\}} sgn(\sigma)\prod^n_{i=1}a_{i,\sigma(i)}$$
 	$$permanent(A) = \sum_{\sigma \in \text{perm}\{1, \cdots, n\}} \prod^n_{i=1}a_{i,\sigma(i)}$$
	
	\noindent
	\textbf{calculating the determinant}\\
	We can convert the matrix into upper triangular form, then we know the determinant is the product of the elements across the diagonal. To zero the $i$ column below the diagonal, we need to do a transposition of columns $O(n)$ and for each of $(n - i + 1) \leq n$ rows below the $i$th row, subtract a multiple of the $i$th row from that row, which contains $n - i + 1 \leq n$ non-zero elements, so $O(n)$ operations per row. So cost of zeroing the $i$th column below the diagonal is $O(n^2)$. There are $n - 1 \leq n$ columns to zero below the diagonal, so cost of converting to UT form is $O(n^3)$ and cost of multiplying diagonal elements is $O(n)$. Hence total cost is $O(n^3 + n) = \bm{O(n^3)}$.\\
	
	\noindent
	\textbf{calculating the permanent}\\
	There are no efficient methods to calculate the permanent - the best-known algorithms run in exponential time.
\end{framed}

\begin{framed}
\begin{center}
	\textbf{\textsc{lexicographic order}}
\end{center}

\noindent
Consider a finite set A which is totally ordered. Given two different elements of the same length $\alpha_1\alpha_2\cdots\alpha_k$ and $\beta_1\beta_2\cdots\beta_k$, the first sequence is smaller than the second one for lexicographic order, if $a_i < b_i$ for the first $i$ where $a_i$ and $b_i$ are different.\\

\noindent
If one sequence is shorter than another, then pad it with "blank" characters - a character than is treated as smaller than every element of $A$.
\end{framed}

\begin{framed}
\begin{center}
	\textbf{\textsc{$\bm{\Omega(n\log n)}$ for comparison-based algorithm}}
\end{center}

\noindent
Suppose we want to sort $n$ elements. There are $n!$ permutations of these $n$ elements. If we draw a binary tree with each leaf represents a permutation of these $n$ elements, the number of comparisons we need at most is the height of tree. A tree with height $h$ has at most $2^h$ leaves, then we have $n! \leq 2^h$, it follows that $log(n!) \leq h$. In the view of $$n! > (\frac{n}{2})^\frac{n}{2} \, \text{for} \, n \geq 1$$ 

\noindent
we know that $h \geq log(n!) \geq log (\frac{n}{2})^\frac{n}{2} = (\frac{n}{2})\, log \, \frac{n}{2}$ so it follows that $h \in \Omega(n \log n)$.
\end{framed}

\begin{framed}
\begin{center}
	\textbf{\textsc{notes}}
\end{center}
\vspace{109mm}
\end{framed}

\end{multicols}
\end{document}