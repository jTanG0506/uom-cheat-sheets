\documentclass[a4paper]{article}

\usepackage[a4paper, margin=0.55in]{geometry}
\usepackage{amssymb}
\usepackage{amsmath}
\usepackage{centernot}
\usepackage{framed}
\usepackage{multicol}

\newcommand*\closure[1]{\overline{#1}}
\newcommand*\interior[1]{{#1}^\circ}
\newcommand*\abs[1]{\vert #1 \vert}
\newcommand*\setremove[2]{#1 \, \backslash \, #2}
\newcommand*\linesep[0]{\noindent\rule{\textwidth}{0.5pt}\\}
\newcommand*\partialfrac[2]{\frac{\partial #1}{\partial #2}}
\newcommand*\e[1]{\text{exp} \, #1}

\begin{document}

\pagenumbering{gobble}

\begin{center}
	\huge{\textbf{MATH20122 Cheat Sheet}}\\
\end{center}

\begin{framed}
	\begin{center}
		\textbf{\textsc{1 Definitions and Examples}}
	\end{center}
	\textbf{metric space}: a \textbf{metric space} $(X, d)$ consists of a non-empty set $X$ and a non-negative real valued \textbf{metric} (\textit{distance function}) $d: X \times X \rightarrow \mathbb{R}^\geq$ which satisfies the following axioms:\\
	\noindent
	(i) $d(x, y) = 0 \iff x = y$ for all $x, y \in X$\\
	(ii) $d(x, y) = d(y, x)$ for all $x, y \in X$\\
	(iii) $d(x, z) \leq d(x, y) + d(y, z)$ for all $x, y, z \in X$ (the \textit{triangle inequality})\\
	
	\noindent
	\textbf{subspace}: given any subset $W \subseteq X$, the restriction of $d$ to $W$ determines the subspace $(W, d := d\vert_W)$ of $(X, d)$\\
	
	\noindent
 	\textbf{open ball}: for any metric space $(X, d)$, the open ball of radius $r > 0$ around any $x \in X$ is $B_r(x) := \{y : d(y, x) < r\}$\\
 	
 	\noindent
 	\textbf{closed ball}: for any metric space $(X, d)$, the open ball of radius $r > 0$ around any $x \in X$ is $\bar{B}_r(x) := \{y : d(y, x) \leq r\}$
 	
 	\linesep
 	\noindent
 	\textbf{euclidean n-space}: $(\mathbb{R}^n, d_2)$ consists of all real \textit{n}-dimensional vectors $x = (x_1, \dots, x_n)$, equipped with the \textbf{euclidean metric} $d_2(x, y) = ((x_1 - y_1)^2 + \cdots + (x_n - y_n)^2)^{1/2}$ where the positive square root is understood\\
 	
 	\noindent
 	\textbf{taxicab metric}: $d_1$ is given on $\mathbb{R}^n$ is given by $d_1(x, y) = \abs{x_1 - y_1} + \cdots + \abs{x_n - y_n}$\\
 	
 	\[ \textbf{discrete metric}: d(x, y) = \begin{cases} 
      0 & \text{if} \; x = y \\
      1 & \text{otherwise}
      \end{cases}
	\]\\
 	
 	\noindent
 	\textbf{isometry}: for any two metric spaces $(X, d_X)$ and $(Y, d_Y)$, a bijection $f: X \rightarrow Y$ is an isometry whenever $d_X(x, y) = d_Y(f(x), f(y))$ for all $x, y \in X$\\
 	
 	\noindent
 	\textbf{standard metric}: $d_\mathbb{C}$ on the complex numbers $\mathbb{C}$ is given by $d_\mathbb{C}(z, z') = \abs{z - z'}$
 	
 	\linesep
 	\noindent
 	\textbf{graph}: $\Gamma := (V, E)$ consists of a set $V$ of \textbf{vertices} and a set $E$ of \textbf{edges}\\
 	
 	\noindent
 	\textbf{path}: a \textbf{path} in $\Gamma$ from $u$ to $w$ is a finite sequence of edges $\pi(u, w) = (uv_1, v_1v_2, \dots, v_{n-2}v_{v-1}, v_{n-1}w)$ with length $n$\\
 	
 	\noindent
 	\textbf{path connected}: a graph is path connected whenever there is a path joining \textit{any} pair of vertices\\
 	
 	\noindent
 	\textbf{edge metric}: $e$ on the vertex set $V$ of a path connected graph is defined by $e(u, w) = min_{\pi(u, w)} l(\pi(u, w))$\\
 	
 	\noindent
 	\textbf{alphabet}: a finite set $A$ of \textbf{letters} and a finite sequence of letters is a \textbf{word} in A. the vertex set $W$ of the associated \textbf{word graph} $\Gamma(A)$ consists of all possible words in A. word $w_1$ and $w_2$ are joined by an edge iff they differ by one of (i) inserting or deleting a letter (ii) swapping two adjacent letters (iii) replacing one letter with another\\
 	
 	\noindent
 	\textbf{word metric}: $d_w$ on $W$ is the edge metric on the associated word graph
 	
 	\linesep
 	\noindent
 	\textbf{binary sequences}: $X = \{0, 1\}^\infty$ is the set of all infinite binary sequences $x = x_0x_1\dots$ where $x_n = 0$ or $1$ for all $n \geq 0$
 	 
 	\[ d_{min}(x, y) = \begin{cases} 
      0 & \text{if} \; x = y \\
      1/2^n & \text{if} \; n = min\{m : x_m \neq y_m\}
      \end{cases}
	\]
	
	$$d^*(x, y) = \sum_{j=0}^\infty \frac{\abs{x_j - y_j}}{2^j}$$
	
	\linesep
	\noindent
	\textbf{bounded}: a real valued function $f$ on a closed interval $[a, b] \subset \mathbb{R}$ is bounded whenever $\exists K, \abs{f(x)} \leq K, \forall x \in [a, b]$\\
	
	\noindent
	let $X$ denote the set of all bounded $f: [a, b] \rightarrow \mathbb{R}$ then
	
	$$d_{sup}(f, g) = \text{sup}_{x \in [a, b]} \abs{f(x) - g(x)} \; \text{with} \; (X, d_{sup}) \; \text{denoted by} \; \mathcal{B}[a, b]$$
	
	\newpage
	\noindent
	let $Y$ denote the set of all continuous $f: [a, b] \rightarrow \mathbb{R}$ then
	
	$$d_1(f, g) = \int_a^b \abs{f(t) - g(t)} dt \; \text{with} \; (Y, d_1) \; \text{denoted by} \; \mathcal{L}_1[0, 1]$$
	
	\linesep
	\noindent
	let $X$ denote the set of all closed intervals $[a, b]$ in the euclidean line\\
	
	\noindent
	\textbf{interval metric}: $d_H$ on $X$ is given by $d_H([a, b], [r, s]) = max\{\abs{r-a}, \abs{s-b}\}$\\
	
	\noindent
	$d_\infty((x_1, x_2), (y_1, y_2)) = max\{\abs{x_1 - y_1}, \abs{x_2 - y_2}\}$
	
	\linesep
	let X be the set of infinite sequences $(a_i : i \geq 0)$ of reals, such that $\sum_i a_i$ is absolutely convergent
	
	$$d_1((a_i), (b_i)) = \sum_{i \geq 0} \abs{a_i - b_i}$$
	
	\linesep
	\textbf{cartesian product}: of two metric spaces $(X, d)$ and $(X', d')$ is the set $X \times X'$ with one of the metrics
	\begin{enumerate}
		\item $d_a((x, x'), (y, y')) = d(x, y) + d'(x', y')$
		\item $d_b((x, x'), (y, y')) = (d(x, y)^2 + d'(x', y')^2)^{1/2}$
		\item $d_c((x, x'), (y, y')) = max\{d(x, y), d'(x', y')\}$
	\end{enumerate}
	
	\linesep
	\noindent
	\textbf{lipschitz equivalent}: two metrics $d$ and $e$ on a given set $X$ are lipschitz equivalent whenever there exists positive constants $h, k \in \mathbb{R}$ such that $he(x, y) \leq d(x, y) \leq ke(x, y)$ for every $x, y \in X$\\
	
	\noindent
	\textbf{theorem}: the metrics $d_a, d_b, d_c$ on $X \times X'$ are lipschitz equivalent
\end{framed}

\begin{framed}
	\begin{center}
		\textbf{\textsc{2 Open and Closed Sets}}
	\end{center}
	in this section, let $(X, d)$ be a any metric space and $U \subseteq X$\\
	
	\noindent
	\textbf{interior point}: $u \in U$ such that $\exists \varepsilon > 0, B_\varepsilon(u) \subseteq U$, the \textbf{interior} of $U$ is the subset $\interior{U} \subseteq U$ of all interior points and if $\interior{U} = U$, then $U$ is \textbf{open} in X\\
	
	\noindent
	\textbf{proposition}: every open ball $B_r(x)$ is open in X\\
	
	\noindent
	\textbf{theorem}: given any two subsets $U, V \subseteq X$, the following hold:
	\begin{multicols}{2}
		\begin{itemize}
			\item $U \subseteq V \implies \interior{U} \subseteq \interior{V}$
			\item $\interior{(\interior{U})} = \interior{U}$
			\item $\interior{U}$ is open in $X$
			\item $\interior{U}$ is the largest subset of $U$, open in $X$
		\end{itemize}	
	\end{multicols}
	
	\noindent
	\textbf{theorem}: the sets $X$ and $\emptyset$ are open in $X$, so are an arbitrary union $U = \bigcup_{i \in I} U_i$ of open sets $U_i$, and a finite intersection $U' = U_1' \cap \cdots \cap U_m'$ of open sets $U_j'$\\
	
	\noindent
	\textbf{closure point}: $x \in X$ is a closure point of $U \subseteq X$ if $B_\varepsilon(x) \cap U$ is non-empty for every $\varepsilon > 0$, the \textbf{closure} of $U$ is the superset $\closure{U} \supseteq U$ of all closure points and if $\closure{U} = U$, then $U$ is \textbf{closed} in $X$\\
	
	\noindent
	\textbf{proposition}: a set $V$ is closed in $X$ iff its complement $U := X \backslash V$ is open\\
	
	\noindent
	\textbf{corollary}: every closed ball $\closure{B}_r(x)$ is closed in $X$\\
	
	\noindent
	\textbf{partially open ball}: in $X$ is a set $P_r(x) := B_r(x) \cup P$, where $P$ is a proper subset of $\{p : d(x, p) = r \}$\\
	
	\noindent
	\textbf{theorem}: given any two subsets $U, V \subseteq X$, the following hold:
	\begin{multicols}{2}
		\begin{itemize}
			\item $U \subseteq V \implies \closure{U} \subseteq \closure{V}$
			\item $\closure{\closure{V}} = \closure{V}$
			\item $\closure{V}$ is closed in $X$
			\item $\closure{V}$ is the smallest set containing $V$, closed in $X$
		\end{itemize}	
	\end{multicols}
	
	\noindent
	\textbf{theorem}: the sets $X$ and $\emptyset$ are closed in $X$, so are an arbitrary intersection $V = \bigcap_{i \in I} V_i$ of closed sets $V_i$ and a finite union $V' = V_1' \cup \cdots \cup V_m'$ of closed sets $V_j'$
	
	\linesep
	\noindent
	\textbf{sequence}: a sequence in $X$ is a function $s: \mathbb{N} \rightarrow X$ and it is standard practise to write $s(n)$ as $x_n$ and display the sequence  as $(x_n : n \geq 1)$\\
	
	\noindent
	\textbf{converges}: a sequence $(x_n)$ converges to the point $x \in X$ whenever $\forall \varepsilon > 0, \exists N \in \mathbb{N},$ such that $n \geq N \implies d(x, x_n) < \varepsilon$ and in this situation, we say that $x$ is the \textbf{limit} of $(x_n)$\\
	
	\noindent
	\textbf{theorem}: in any metric space $(X, d)$, the limit of a convergent sequence is unique\\
	
	\noindent
	\textbf{theorem}: suppose that $Y \subseteq X$ and $y \in X$, then $y$ lies in $\closure{Y}$ iff there exists a sequence $(y_n)$ in $Y$ such that $y_n \rightarrow y$ as $n \rightarrow \infty$\\
	
	\noindent
	\textbf{cauchy sequence}: in any metric space $(X, d)$, a cauchy sequence $(x_n)$ satisfies $\forall \varepsilon < 0, \exists N \in \mathbb{N}$, such that $m, n \geq N \implies d(x_n, x_m) < \varepsilon$\\
	
	\noindent
	\textbf{dense}: a subset $Y$ is dense in $(X, d)$ whenever $\closure{Y} = X$
	
	\linesep
	\textbf{bounded}: a subset $A$ of the metric space $(X,d)$ is bounded whenever there exists $x_0 \in X$ and $M \in \mathbb{R}$ such that $d(x, x_0) \leq M$ for every $x \in A$. a function $f: S \rightarrow X$ is bounded whenever its image $f(S) \subset X$ is a bounded, for any set\\
	
	\noindent
	\textbf{diameter}: the diameter, $diam(A)$ of a bounded non-empty set $A \subseteq X$ is the real number $sup\{d(x, y) : x, y \in A\}$\\
	
	\noindent
	\textbf{eucildean (n - 1)-sphere of radius r}: $\{x : \abs{x} = r\}$\\
	
	\noindent
	\textbf{boundary point}: $x \in X$ of $A$ is one for which every open ball $B_\varepsilon(x)$ meets both $A$ and $\setremove{X}{A}$, the \textbf{boundary} $\delta A$ of $A$ is the set of all such boundary points\\
	
	\noindent
	\textbf{theorem}: any subset $A$ of $(X, d)$ satisfies:
		\begin{multicols}{3}
		\begin{itemize}
			\item $\setremove{A}{\delta A} = \setremove{\closure{A}}{\delta A} = \interior{A}$
			\item $\delta A = \delta(\setremove{X}{A})$
			\item $\delta A$ is closed in $X$
		\end{itemize}	
	\end{multicols}
\end{framed}

\end{document}