\documentclass[a4paper]{article}

\usepackage[a4paper, margin=0.55in]{geometry}
\usepackage{amssymb}
\usepackage{amsmath}
\usepackage{centernot}
\usepackage{framed}
\usepackage{multicol}

\newcommand*\conj[1]{\overline{#1}}
\newcommand*\abs[1]{\vert #1 \vert}
\newcommand*\setremove[2]{#1 \, \backslash \, #2}
\newcommand*\linesep[0]{\noindent\rule{\textwidth}{0.5pt}\\}
\newcommand*\partialfrac[2]{\frac{\partial #1}{\partial #2}}
\newcommand*\e[1]{\text{exp} \, #1}

\begin{document}

\pagenumbering{gobble}

\begin{center}
	\huge{\textbf{MATH20142 Cheat Sheet}}\\
\end{center}

\begin{framed}
	\begin{center}
		\textbf{\textsc{1 Construction and Basic Properties of Complex Numbers}}
	\end{center}
	
	\noindent
	An expression $a + ib (a, b \in \mathbb{R})$ is called a \textbf{complex number}. We denote the set of complex numbers by $\mathbb{C}$. For $z = x + iy$, we use $x = \text{Re}z$ and $y = \text{Im}z$ and say that $z$ is real if $\text{Im}z = 0$ and that $z$ is imaginary if $\text{Re}z = 0$.
	
	\begin{multicols}{3}
		\begin{itemize}
			\item $\text{Re}(z \pm w) = \text{Re}z \pm \text{Re}w$
			\item $\text{Im}(z \pm w) = \text{Im}z \pm \text{Re}w$
			\item $\conj{(z \pm w)} = \conj{z} \pm \conj{w}$
			\item $\conj{zw} = \conj{z} \, \conj{w}$
			\item $\conj{(z/w)} = \conj{z}/\conj{w}$ if $w \neq 0$
			\item $z + \conj{z} = 2\text{Re}z$
			\item $z - \conj{z} = 2\text{Im}z$
			\item $ \abs{z} = 0 \iff z = 0$
			\item $ \abs{zw} = \abs{z} \abs{w}$
			\item $ \abs{z / w} = \abs{z} / \abs{w}$ if $w \neq 0$
			\item $\abs{z + w} \leq \abs{z} + \abs{w}$
			\item $\abs{z - w} \geq \abs{\abs{z} - \abs{w}}$
		\end{itemize}
	\end{multicols}
\end{framed}

\begin{framed}
	\begin{center}
		\textbf{\textsc{2 Topology in $\mathbb{C}$}}
	\end{center}
	$\varepsilon$\textbf{-neighbourhood of} $z_0$: $N_\varepsilon(z_0) = \{z \in \mathbb{C} : \abs{z - z_0} < \varepsilon \}$ (disc centred at $z_0$ containing points with distance $< \varepsilon$)\\
	
	\noindent
	\textbf{limit point}: $z_0 \in \mathbb{C}$ is a limit point of a set $S \subset \mathbb{C}$ if, for every $\varepsilon > 0, N_\varepsilon(z_0)$ contains a point in $\setremove{S}{\{z_0\}}$\\
	
	\noindent
	\textbf{interior point}: let $S \subset C, z_0$ a limit point of $S$, then $z_0$ is an interior point of S if $\exists \, \varepsilon > 0, N_\varepsilon (z_0) \subset S$\\
	
	\noindent
	\textbf{boundary point}: let $S \subset C, z_0$ a limit point of $S$, then $z_0$ is an boundary point of S if it is not a interior point\\
	
	\noindent
	\textbf{open}: a set $S \subset \mathbb{C}$ is called open if it consists only of interior points\\
	
	\noindent
	\textbf{domain}: let $S \subset \mathbb{C}, S \neq \emptyset$, then S is called a domain if S is open and every pair of points can be connected by a polygonal arc lying entirely in S\\
	
	\noindent
	\textbf{function}: let $S \subset C, S \neq \emptyset$, a function $f: \subset \rightarrow \mathbb{C}$ is a rule which assigns to each $z \in S$, an image $f(z) \in \mathbb{C}$\\
	
	\noindent
	\textbf{$\lim_{z \rightarrow z_0} f(z)$}: let $f: S \rightarrow \mathbb{C}$ be a function. if $z_0$ is a limit point of $S$ then we say $\lim_{z \rightarrow z_0} f(z) = l$ if, $\forall \varepsilon > 0, \exists \delta > 0, s \in S$ and $0 < \abs{z - z_0} < \delta \implies \abs{f(z) - l} < \varepsilon$\\
	
	\noindent
	\textbf{continuity}: $f(z)$ is continuous at $z_0$ if $\lim_{z \rightarrow z_0} f(z) = f(z_0)$\\
	
	\linesep
	
	\noindent
	\textbf{proposition} a set $S \subset \mathbb{C}$ is closed $\iff$ its complement $\setremove{\mathbb{C}}{S}$ is open\\
	
	\noindent
	\textbf{proposition} if $\lim_{z \rightarrow z_0} f(z) = l$ and $\lim_{z \rightarrow z_0} g(z) = k$, then
	\begin{enumerate}
		\item $\lim_{z \rightarrow z_0} (f(z) \pm g(z)) = l \pm k$
		\item $\lim_{z \rightarrow z_0} (f(z)g(z)) = lk$
		\item $\lim_{z \rightarrow z_0} (f(z)/g(z)) = l/k$ (for $k \neq 0$)
	\end{enumerate}
	
	\noindent
	\textbf{proposition} $\lim_{z \rightarrow z_0} f(z) = l = \alpha + i\beta \, (\alpha, \beta \in \mathbb{R}) \iff u(x, y) \rightarrow \alpha, v(x, y) \rightarrow \beta$, as $(x, y) \rightarrow (\text{Re}z, \text{Im}z)$
\end{framed}

\begin{framed}
	\begin{center}
		\textbf{\textsc{3 Differentiation and Cauchy-Riemann Equations}}
	\end{center}
	\textbf{differentiable at a point}: let $S \subset \mathbb{C}$ be a open set. we say that $f: S \rightarrow \mathbb{C}$ is differentiable at a point $z_0 \in S$ with derivative $f'(z_0)$ if
	$$\lim_{z \rightarrow z_0}\frac{f(z) - f(z_0)}{z - z_0} = f'(z_0)$$

	\noindent
	\textbf{differentiable function}: if $f$ is differentiable at every point of $S$, we say $f$ is a differentiable function in S
	
	\noindent
	\textbf{partial derivatives}: for $z = x + iy$, write $f(z) = u(x, y) + iv(x, y)$, where $u, v$ are real-valued
	
	\begin{align*}
  		u_x = \partialfrac{u}{x} &= \lim_{h \rightarrow 0}\frac{u(x + h, y) - u(x, y)}{h} & v_x = \partialfrac{v}{x} &= \lim_{h \rightarrow 0}\frac{v(x + h, y) - v(x, y)}{h}\\
     	u_ y = \partialfrac{u}{y} &= \lim_{k \rightarrow 0}\frac{u(x, y + k) - u(x, y)}{k} & v_y = \partialfrac{v}{y} &= \lim_{k \rightarrow 0}\frac{v(x, y + k) - v(x, y)}{k}
	\end{align*}
	
	\linesep
	
	\noindent
	\textbf{proposition} if $f$ is differentiable at $z_0$ then $f$ is continuous at $z_0$\\
	
	\noindent
	\textbf{proposition} if $f$ is differentiable at $z = x + iy$ then $u_x, u_y, v_x, v_y$ all exist and $u_x = v_y, v_x = -u_y$ (CRE)\\
	
	\noindent
	\textbf{theorem} if $f(z) = u(x, y) + iv(x, y)$ is a complex function on an open set $S$ and at $z_0 = x_0 + iy_0 \in S$, the partial derivatives $u_x, v_x, u_y, v_y$ all exist, are continuous and satisfy the CRE then $f$ is differentiable at $z_0$\\
	
	\noindent
	\textbf{theorem} if $f$ is differentiable in a domain $D$ and $f'(z) = 0$ for all $z \in D$, then $f$ is constant in $D$
\end{framed}

\begin{framed}
	\begin{center}
		\textbf{\textsc{4 Power Series}}
	\end{center}
	\textbf{convergence}: we say a sequence $s_n \in \mathbb{C}$ converges to $s \in \mathbb{C}$ if, $\forall \varepsilon > 0, \exists N \in \mathbb{N}$ such that $\abs{s_n - s} < \varepsilon, \forall n \geq N$. the series $\sum_{k=0}^\infty z_k$ converges if the sequence of partial sums $s_n = \sum_{k=0}^n z_k$ converges and the limit of the sequence is called the sum of the series\\
	
	\noindent
	\textbf{divergent series}: a series which does not converge is said to be divergent\\
	
	\noindent
	\textbf{absolute convergence}: we say that $\sum_{k=0}^\infty z_k$ is absolutely convergent if the real series $\sum_{k=0}^\infty \abs{z_k}$ is convergent\\
	
	\noindent
	\textbf{ratio test}: consider $\sum_{k=0}^\infty z_k$ and suppose that $\lim_{n \rightarrow \infty}\abs{z_{n+1}} / \abs{z_n} = l$. if $l < 1$ then $\sum_{k=0}^\infty z_k$ is absolutely convergent and if $l > 1$ then $\sum_{k=0}^\infty z_k$ diverges\\
	
	\noindent
	\textbf{root test}: consider $\sum_{k=0}^\infty z_k$ and suppose that $\lim_{n \rightarrow \infty}\abs{z}^{1/n} = l$. if $l < 1$ then $\sum_{k=0}^\infty z_k$ is absolutely convergent and if $l > 1$ then $\sum_{k=0}^\infty z_k$ diverges\\
	
	\noindent
	\textbf{general principle of convergence}: if a series $\sum_{n=1}^\infty s_n$ with $s_n \in \mathbb{C}$ converges, then $s_n \rightarrow 0$ as $n \rightarrow \infty$\\
	
	\noindent
	\textbf{power series about} $z_0$: $\sum_{n=0}^\infty a_nz^n$\\
	
	\noindent
	\textbf{radius of convergence}: $R = \text{sup}\{r : \exists z \; \text{such that} \; \abs{z} = r \; \text{and} \; \sum_{n=0}^\infty a_nz^n \; \text{converges} \; \}$\\
	
	\noindent
	\textbf{disc of convergence}: $\{ z \in \mathbb{C} : \abs{z} < R\}$, where $R$ is the radius of convergence\\
	
	\noindent
	\textbf{computation of radius of convergence}: $R = \lim_{n \rightarrow \infty} \abs{a_{n-1} / a_n}$, provided the limit exists\\	
	
	\linesep
	
	\noindent
	\textbf{lemma} if a power series $\sum_{n=0}^\infty a_nz^n$ converges for $z = z_1 \neq 0$, then it converges absolutely for all $z$ with $\abs{z} < \abs{z_1}$\\
	
	\noindent
	\textbf{lemma} if $\sum_{n=0}^\infty a_nz^n$ diverges for $z = z_2$, then is diverges for all $z$ with $\abs{z} > \abs{z_2}$\\
	
	\noindent
	\textbf{theorem} the radius of convergence $R$ of $\sum_{n=0}^\infty a_nz^n$ is given by $1/R = \lim_{n \rightarrow \infty} \abs{a_n}^{1/n}$\\
	
	\noindent
	\textbf{lemma} if $f(z) = \sum_{n=0}^\infty a_nz^n$ converges absolutely for $\abs{z} < R$ then $g(z) = \sum_{n=0}^\infty na_nz^{n-1}$ converges for $\abs{z} < R$\\
	
	\noindent
	\textbf{theorem} a power series $f(z) = \sum_{n=0}^\infty a_nz^n$ may be differentiated term by term within its disc of convergence so that $f'(z) = \sum_{n=0}^\infty na_nz^{n-1}$\\
	
	\noindent
	\textbf{corollary} all higher derivatives $f', f'', f''', \dots, f^{(n)}, \dots$ of a power series $f(z) = \sum_{n=0}^\infty a_nz^n$ exist for $z$ within the disc of convergence and $f^{(k)}(z) = \sum_{n=k}^\infty n(n - 1)\cdots(n - k + 1)a_nz^{n - k} = \sum_{n=k}^\infty n!/(n - k)! \cdot a_nz^{n - k}$\\
	
	\noindent
	\textbf{corollary} if $f(z) = \sum_{n=0}^\infty a_n(z - z_0)^n$ has disc of convergence $\abs{z - z_0} < R$ then $a_k = f^{(k)}(z_0)/k!$ and we can express $f$ as a \textbf{Taylor series} $f(z) = \sum_{n=0}^\infty f^{(n)}(z_0)/n! \cdot (z - z_0)^n$, valid for $\abs{z - z_0} < R$
\end{framed}

\begin{framed}
	\begin{center}
		\textbf{\textsc{5 The Exponential Function and Its Friends}}
	\end{center}
	\textbf{the exponential function}: define $\e{z} = \sum_{n=0}^\infty z^n / n!$, which converges absolutely for all $z \in \mathbb{C}$. we can check that $\e{(z_1 + z_2)} = \e{z_1}\e{z_2}$ and by induction, $\e{nz} = (\e{z})^n$, for all $n \in \mathbb{Z}^+$\\
	
	\noindent
	\textbf{the number $e$}: define $e = \e{1} = 2.7182818 \dots$ and we can also use the notation $\e{z} = e^z$, for all $z \in \mathbb{C}$\\
	
	\noindent
	\textbf{trigonometric functions}: define $\cos z = \sum_{n=0}^\infty (-1)^n z^{2n} / (2n)!$ and $\sin z = \sum_{n=0}^\infty (-1)^n z^{2n + 1} / (2n + 1)!$\\
	
	\noindent
	\textbf{hyperbolic functions}: define $\cosh z = \frac{1}{2}(e^z + e^{-z})$ and $\sinh z = \frac{1}{2}(e^z - e^{-z})$ so $\sin iz = i\sinh z$ and $\cos iz = \cosh z$\\
	
	\noindent
	\textbf{period}: for a function $f: \mathbb{C} \rightarrow \mathbb{C}$, a nonzero number $k \in \mathbb{C}$ is called a period if $f(z + k) = f(z)$, for all $z \in \mathbb{C}$\\
	
	\noindent
	\textbf{logarithmic function}: $\log z = u + iv = log \abs{z} + i \arg z$ and $\text{Log} z = \log z + i \arg z$ ($-\pi < \arg z \leq \pi$)\\
	
	\noindent
	\textbf{cut plane}: the complex plane with the negative real axis, including zero, removed is called the cut plane and denoted $\mathbb{C}_\pi$
	
	\linesep
	
	\noindent
	\textbf{eulers theorem}: $e^{iz} = \cos z + i \sin z$\\
	
	\noindent
	\textbf{corollary}
		\begin{multicols}{3}
		\begin{itemize}
			\item $\cos z = \frac{1}{2}(e^{iz} + e^{-iz})$
			\item $\sin z = \frac{1}{2i}(e^{iz} - e^{-iz})$
			\item $\cos^2 z + \sin^2 z = 1$
		\end{itemize}
	\end{multicols}
	\begin{multicols}{2}
		\begin{itemize}
			\item $\sin(z + w) = \sin z \cos w + \cos z \sin w$
			\item $\cos(z + w) = \cos z \cos w - \sin z \sin w$
		\end{itemize}
	\end{multicols}
	
	\noindent
	\textbf{lemma}: the functions $arg(z)$ and $Log(z)$ are continuous on the cut plane\\
	
	\noindent
	\textbf{theorem}: let $z \neq 0$ be a complex and let $n$ be a positive integer, then $$z^{\frac{1}{n}} = \{ \abs{z}^{\frac{1}{n}} e^{i (\frac{Argz + 2k\pi}{n})} \; \vert \; k = 0, 1, \dots, n - 1 \}$$
\end{framed}

\begin{framed}
	\begin{center}
		\textbf{\textsc{6 Integration}}
	\end{center}
	\textbf{path}: a path is a function $\gamma: [a, b] \rightarrow \mathbb{C}$, where $[a, b]$ is a real interval\\
	
	\noindent
	\textbf{closed path}: $\gamma$ is a closed path if $\gamma(a) = \gamma(b)$ (it starts and ends at the same point)\\
	
	\noindent
	\textbf{smooth path}: a path $\gamma$ is smooth if $\gamma: [a, b] \rightarrow \mathbb{C}$ is differentiable and $\gamma'$ is continuous (one-sided derivatives at $a$ and $b$)\\
	
	\noindent
	\textbf{length of a path}: $L(\gamma) = \int_a^b \abs{\gamma'(t)} \; dt$\\
	
	\noindent
	\textbf{contour}: a contour is a collection of smooth paths $\gamma_1, \dots, \gamma_n$ where the end point of $\gamma_r$ coincides with the start point of $\gamma_{r+1}$ for $r = 1, \dots, n - 1$. if the end point of $\gamma_n$ coincides with the start point of $\gamma_1$, then $\gamma$ is a closed contour.\\
	
	\noindent
	\textbf{length of contour}: $\gamma = \gamma_1 + \dots + \gamma_n$ is $L(\gamma) = L(\gamma_1) + \dots + L(\gamma_n)$\\
	
	\noindent
	\textbf{opposite path}: if $\gamma: [a, b] \rightarrow \mathbb{C}$ is path then $-\gamma: [b, a] \rightarrow \mathbb{C}$ defined by $-\gamma(t) = \gamma(a + b - t)$ is called the opposite path\\
	
	\noindent
	\textbf{the integral of $f$ along $\gamma$}: $\int_\gamma f(z) \; dz = \int_a^b f(\gamma(t))\gamma'(t) \; dt = \int_a^b U(t) \; dt + i \int_a^b V(t) \; dt$ where $U, V: [a, b] \rightarrow \mathbb{R}$\\
	
	\noindent
	\textbf{winding number of $\gamma$ around $z_0$}: $w(\gamma, z_0)$, the number of times $\gamma$ winds around $z_0$, with anticlockwise as $+ve$\\
	
	\noindent
	\textbf{simply connected}: a domain $D$ is simply connected if $w(\gamma, z) = 0$ for every closed contour $\gamma$ in $D$ and $z \notin D$\\
	
	\noindent
	\textbf{analytic}: a function $f: D \rightarrow \mathbb{C}$ is called analytic if it can be expanded into a Taylor series around any point in $D$\\
	
	\noindent
	\textbf{bounded}: we say that a function $f: D \rightarrow \mathbb{C}$ is bounded if there exists $M \geq 0$ such that $\abs{f(z)} \leq M$ for all $z \in \mathbb{C}$
	
	\linesep
	
	\noindent
	\textbf{properties of contour integration}
	\begin{multicols}{2}
		\begin{itemize}
			\item $\int_{\gamma_1 + \gamma_2} f = \int_{\gamma_1} f + \int_{\gamma_2} f$
			\item $\int_\gamma (f_1 + f_2) = \int_\gamma f_1 + \int_\gamma f_2$
			\item $\int_\gamma cf = c \int_\gamma f$
			\item $\int_{-\gamma} f = - \int_\gamma f$
		\end{itemize}
	\end{multicols}
	
	\noindent
	\textbf{fundamental theorem of contour integration}: if $f: D \rightarrow \mathbb{C}$ is continuous, $F: D \rightarrow \mathbb{C}$ satisfies $F' = f$ and $\gamma$ is a contour in $D$ from $z_0$ to $z_1$, then $\int_\gamma f = F(z_1) - F(z_0)$\\
	
	\noindent
	\textbf{cauchys theorem}: let $f$ be differentiable in a domain $D$ and $\gamma$ a closed contour in $D$ which does not wind around any point outside $D$, then $\int_\gamma f = 0$\\
	
	\noindent
	\textbf{generalised cauchys theorem}: suppose that $\gamma_1, \dots, \gamma_n$ are closed contour in a domain such that $w(\gamma_1, z) + \dots + w(\gamma_n, z) = 0, \forall z \notin D$. if $f$ is differentiable in $D$ then $\int_{\gamma_1} f + \cdots + \int_{\gamma_n} f = 0$\\
	
	\noindent
	\textbf{corollary}: let $f$ be differentiable in a simply connected domain $D$ and let $\gamma$ a closed contour in $D$, then $\int_\gamma f = 0$\\
	
	\noindent
	\textbf{cauchys integral formula for a circle}: let $f$ be differentiable in the disc $\{ z \in \mathbb{C} : \abs{z - z_0} < R \}$. for $0 < r < R$, let $C_r$ be the path $C_r(t) = z_0 + re^{it}$, $0 \leq t \leq 2\pi$ then for $\abs{w - z_0} < r$, $f(w) = \frac{1}{2\pi i} \int_{C_r} \frac{f(z)}{z - w} \; dz$\\
	
	\noindent
	\textbf{theorem}: if $f$ is a differentiable in a domain $D$, then all the higher derivatives of $f$ exist in $D$ and, for any disc $\{ z \in \mathbb{C} : \abs{z - z_0} < R \}$, $f$ has a Taylor series expansion $f(z) = \sum_{n=0}^\infty \frac{f^{(n)}(z_0)}{n!}(z - z_0)^n$\\
	
	\noindent
	\textbf{the estimation lemma}: let $D$ be a domain. if $f: D \rightarrow \mathbb{C}$ is continuous, $\gamma$ is a contour in $D$, $\abs{f(z)} \leq M$ for all $z$ on $\gamma$, then $\abs{\int_\gamma f} \leq M \cdot L(\gamma)$\\
	
	\noindent
	\textbf{cauchys estimate}: suppose that $f$ is differentiable in $\{ z \in \mathbb{C} : \abs{z - z_0} < R \}$. if $0 < r < R$ and $\abs{f(z)} \leq M$ for $\abs{z - z_0} = r$, then for all $n \geq 0, \abs{f^{(n)}(z_0)} \leq \frac{Mn!}{r^n}$\\
	
	\noindent
	\textbf{liouvilles theorem}: if $f$ if differentiable and bounded in the whole complex plane then $f$ is constant\\
	
	\noindent
	\textbf{corollary}: suppose $f: \mathbb{C} \rightarrow \mathbb{C}$ is differentiable in of $\mathbb{C}$ and there exists $C > 0$ such that $\abs{f(z)} \leq C \abs{z}$, $\forall z \in \mathbb{C}$, then $f(z) = az$ for some $a \in \mathbb{C}$\\
	
	\noindent
	\textbf{fundamental theorem of algebra}: let $P(z) = z^n + a_1z^{n-1} + \dots + a_{n-1}z + a_n$ be a polynomial with $n \geq 1$ and $a_1, \dots, a_n \in \mathbb{C}$, then there exists $w \in \mathbb{C}$ with $P(w) = 0$\\
	
	\noindent
	\textbf{corollary}: each polynomial of degree $n$ with complex coefficients has exactly $n$ complex roots, taken with their multiplicity
\end{framed}

\begin{framed}
	\begin{center}
		\textbf{\textsc{7 Laurent Series}}
	\end{center}
	\textbf{isolated singularities}: if $f$ is differentiable in a punctured disc $0 < \abs{z - z_0} < R$ then we say that $z_0$ is a isolated singularity of $f$. such $f$ has a Laurent expansion: $f(z) = \sum_{n = 0}^\infty a_n(z - z_0)^n + \sum_{n = 1}^\infty b_n(z - z_0)^{-n}$, for $\abs{z - z_0} < R$
	
	\begin{enumerate}
		\item $b_n = 0$ for all $n \geq 1$. if we define $f(z_0) = a_0$, we obtain a function which is differentiable in the whole disc $\abs{z - z_0} < R$, with Taylor series $\sum_{n = 0}^\infty a_n (z - z_0)^n$. in this case, we say that $z_0$ is a \textbf{removable singularity}
		\item only finitely many $b_n$ are non-zero, then we can write $f(z) = \frac{b_m}{(z - z_0)^m} + \cdots + \frac{b_1}{z - z_0} + \sum_{n=0}^\infty a_n (z - z_0)^n$ where $b_m \neq = 0$. in this case, we say that $f$ has a \textbf{pole of order m} at $z_0$. a pole of order one is called a \textbf{simple pole}
		\item infinitely many $b_n$ are non-zero. then we say that $z_0$ is an \textbf{isolated essential singularity}
	\end{enumerate}
	
	\linesep
	
	\noindent
	\textbf{laurents theorem}: if $f$ is differentiable in the annulus $\{ z \in \mathbb{C} : R_1 \leq \abs{z - z_0} \leq R_2 \}$ where $0 \leq R_1 \leq R_2 \leq \infty$ then $f(z) = \sum_{n=0}^\infty a_n(z - z_n)^n + \sum_{n=1}^\infty b_n(z - z_n)^{-n}$, where $\sum_{n=0}^\infty a_n(z - z_n)^n$ converges for $\abs{z - z_0} < R_2$ and $\sum_{n=1}^\infty b_n(z - z_n)^{-n}$ converges for $\abs{z - z_0} > R_1$. in particular, both series converge in $\{ z \in \mathbb{C} : R_1 < \abs{z - z_0} < R_2 \}$\\
	
	\noindent
	furthermore, if $C_R(t) = z_0 + re^{it}$ with $R_1 < r < R_2$, $0 \leq t \leq 2\pi$ then $$a_n = \frac{1}{2\pi i} \int_{C_r} \frac{f(z)}{(z - z_0)^{n + 1}} dz \quad \text{and} \quad b_n = \frac{1}{2\pi i} \int_{C_r} f(z)(z - z_0)^{n - 1} dz$$
\end{framed}


\end{document}