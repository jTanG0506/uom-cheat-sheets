\documentclass[a4paper]{article}

\usepackage[a4paper, margin=0.55in]{geometry}
\usepackage{amssymb}
\usepackage{amsmath}
\usepackage{centernot}
\usepackage{framed}
\usepackage{multicol}

\newcommand*\conj[1]{\overline{#1}}
\newcommand*\abs[1]{\vert #1 \vert}
\newcommand*\setremove[2]{#1 \, \backslash \, #2}
\newcommand*\linesep[0]{\noindent\rule{\textwidth}{0.5pt}\\}
\newcommand*\partialfrac[2]{\frac{\partial #1}{\partial #2}}

\begin{document}

\pagenumbering{gobble}

\begin{center}
	\huge{\textbf{MATH20142 Cheat Sheet}}\\
\end{center}

\begin{framed}
	\begin{center}
		\textbf{\textsc{1 Construction and Basic Properties of Complex Numbers}}
	\end{center}
	
	\noindent
	An expression $a + ib (a, b \in \mathbb{R})$ is called a \textbf{complex number}. We denote the set of complex numbers by $\mathbb{C}$. For $z = x + iy$, we use $x = \text{Re}z$ and $y = \text{Im}z$ and say that $z$ is real if $\text{Im}z = 0$ and that $z$ is imaginary if $\text{Re}z = 0$.
	
	\begin{multicols}{3}
		\begin{itemize}
			\item $\text{Re}(z \pm w) = \text{Re}z \pm \text{Re}w$
			\item $\text{Im}(z \pm w) = \text{Im}z \pm \text{Re}w$
			\item $\conj{(z \pm w)} = \conj{z} \pm \conj{w}$
			\item $\conj{zw} = \conj{z} \, \conj{w}$
			\item $\conj{(z/w)} = \conj{z}/\conj{w}$ if $w \neq 0$
			\item $z + \conj{z} = 2\text{Re}z$
			\item $z - \conj{z} = 2\text{Im}z$
			\item $ \abs{z} = 0 \iff z = 0$
			\item $ \abs{zw} = \abs{z} \abs{w}$
			\item $ \abs{z / w} = \abs{z} / \abs{w}$ if $w \neq 0$
			\item $\abs{z + w} \leq \abs{z} + \abs{w}$
			\item $\abs{z - w} \geq \abs{\abs{z} - \abs{w}}$
		\end{itemize}
	\end{multicols}
\end{framed}

\begin{framed}
	\begin{center}
		\textbf{\textsc{2 Topology in $\mathbb{C}$}}
	\end{center}
	$\varepsilon$\textbf{-neighbourhood of} $z_0$: $N_\varepsilon(z_0) = \{z \in \mathbb{C} : \abs{z - z_0} < \varepsilon \}$ (disc centred at $z_0$ containing points with distance $< \varepsilon$)\\
	
	\noindent
	\textbf{limit point}: $z_0 \in \mathbb{C}$ is a limit point of a set $S \subset \mathbb{C}$ if, for every $\varepsilon > 0, N_\varepsilon(z_0)$ contains a point in $\setremove{S}{\{z_0\}}$\\
	
	\noindent
	\textbf{interior point}: let $S \subset C, z_0$ a limit point of $S$, then $z_0$ is an interior point of S if $\exists \, \varepsilon > 0, N_\varepsilon (z_0) \subset S$\\
	
	\noindent
	\textbf{boundary point}: let $S \subset C, z_0$ a limit point of $S$, then $z_0$ is an boundary point of S if it is not a interior point\\
	
	\noindent
	\textbf{open}: a set $S \subset \mathbb{C}$ is called open if it consists only of interior points\\
	
	\noindent
	\textbf{domain}: let $S \subset \mathbb{C}, S \neq \emptyset$, then S is called a domain if S is open and every pair of points can be connected by a polygonal arc lying entirely in S\\
	
	\noindent
	\textbf{function}: let $S \subset C, S \neq \emptyset$, a function $f: \subset \rightarrow \mathbb{C}$ is a rule which assigns to each $z \in S$, an image $f(z) \in \mathbb{C}$\\
	
	\noindent
	\textbf{$\lim_{z \rightarrow z_0} f(z)$}: let $f: S \rightarrow \mathbb{C}$ be a function. if $z_0$ is a limit point of $S$ then we say $\lim_{z \rightarrow z_0} f(z) = l$ if, $\forall \varepsilon > 0, \exists \delta > 0, s \in S$ and $0 < \abs{z - z_0} < \delta \implies \abs{f(z) - l} < \varepsilon$\\
	
	\noindent
	\textbf{continuity}: $f(z)$ is continuous at $z_0$ if $\lim_{z \rightarrow z_0} f(z) = f(z_0)$\\
	
	\linesep
	
	\noindent
	\textbf{proposition} a set $S \subset \mathbb{C}$ is closed $\iff$ its complement $\setremove{\mathbb{C}}{S}$ is open\\
	
	\noindent
	\textbf{proposition} if $\lim_{z \rightarrow z_0} f(z) = l$ and $\lim_{z \rightarrow z_0} g(z) = k$, then
	\begin{enumerate}
		\item $\lim_{z \rightarrow z_0} (f(z) \pm g(z)) = l \pm k$
		\item $\lim_{z \rightarrow z_0} (f(z)g(z)) = lk$
		\item $\lim_{z \rightarrow z_0} (f(z)/g(z)) = l/k$ (for $k \neq 0$)
	\end{enumerate}
	
	\noindent
	\textbf{proposition} $\lim_{z \rightarrow z_0} f(z) = l = \alpha + i\beta \, (\alpha, \beta \in \mathbb{R}) \iff u(x, y) \rightarrow \alpha, v(x, y) \rightarrow \beta$, as $(x, y) \rightarrow (\text{Re}z, \text{Im}z)$
\end{framed}

\begin{framed}
	\begin{center}
		\textbf{\textsc{3 Differentiation and Cauchy-Riemann Equations}}
	\end{center}
	\textbf{differentiable at a point}: let $S \subset \mathbb{C}$ be a open set. we say that $f: S \rightarrow \mathbb{C}$ is differentiable at a point $z_0 \in S$ with derivative $f'(z_0)$ if
	$$\lim_{z \rightarrow z_0}\frac{f(z) - f(z_0)}{z - z_0} = f'(z_0)$$

	\noindent
	\textbf{differentiable function}: if $f$ is differentiable at every point of $S$, we say $f$ is a differentiable function in S
	
	\noindent
	\textbf{partial derivatives}: for $z = x + iy$, write $f(z) = u(x, y) + iv(x, y)$, where $u, v$ are real-valued
	
	\begin{align*}
  		u_x = \partialfrac{u}{x} &= \lim_{h \rightarrow 0}\frac{u(x + h, y) - u(x, y)}{h} & v_x = \partialfrac{v}{x} &= \lim_{h \rightarrow 0}\frac{v(x + h, y) - v(x, y)}{h}\\
     	u_ y = \partialfrac{u}{y} &= \lim_{k \rightarrow 0}\frac{u(x, y + k) - u(x, y)}{k} & v_y = \partialfrac{v}{y} &= \lim_{k \rightarrow 0}\frac{v(x, y + k) - v(x, y)}{k}
	\end{align*}
	
	\linesep
	
	\noindent
	\textbf{proposition} if $f$ is differentiable at $z_0$ then $f$ is continuous at $z_0$\\
	
	\noindent
	\textbf{proposition} if $f$ is differentiable at $z = x + iy$ then $u_x, u_y, v_x, v_y$ all exist and $u_x = v_y, v_x = -u_y$ (CRE)\\
	
	\noindent
	\textbf{theorem} if $f(z) = u(x, y) + iv(x, y)$ is a complex function on an open set $S$ and at $z_0 = x_0 + iy_0 \in S$, the partial derivatives $u_x, v_x, u_y, v_y$ all exist, are continuous and satisfy the CRE then $f$ is differentiable at $z_0$\\
	
	\noindent
	\textbf{theorem} if $f$ is differentiable in a domain $D$ and $f'(z) = 0$ for all $z \in D$, then $f$ is constant in $D$
\end{framed}


\end{document}