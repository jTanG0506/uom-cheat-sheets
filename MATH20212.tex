\documentclass[a4paper]{article}

\usepackage[a4paper, margin=0.55in]{geometry}
\usepackage{amssymb}
\usepackage{amsmath}
\usepackage{centernot}
\usepackage{framed}

\begin{document}

\pagenumbering{gobble}

\begin{center}
	\huge{\textbf{MATH20212 Cheat Sheet}}\\
\end{center}

\begin{framed}
	\begin{center}
		\textbf{\textsc{1 Rings}}
	\end{center}
	A \textbf{ring} is a set $R$ and two binary operations, written $+ \, \text{and} \, \times$, on R which satisfies the following conditions:\\
	(R1) $\langle R, + \rangle$ is an abelian group with identity 0\\
	(R2) $\times$ is associative\\
	(R3) $\times$ is distributive over $+$\\
	(R4) there exists an element $1 \in R$, different from 0, that is an identity for $\times$\\
	
	\noindent
	Let $R$ be a ring and $S \subseteq R$. Then $S$ is a \textbf{subring} of R if it is a ring in its own right with respect to the same addition and multiplication as in $R$ and $S$ contains $1_R$.\\
	
	\noindent
	\textbf{Subring Test}: Let $R$ be a ring and $S \subseteq R$, then $S$ is a subring of $R$, iff:\\
	$(i) \, 1 \in S$\\
	$(ii) \, r + s, r \times s \in S$, for all $r, s \in S$\\
	$(iii) \, -r \in S$ for all $r \in S$\\
	
	\noindent
	Let R be a ring. The \textbf{ring of polynomials} $R[X]$ in the indeterminate $X$ is defined as follows:\\
	\textbf{Elements}: formal linear combinations of the form $\sum_{i \geq 0}a_iX^i$ with $a_i \in R$ for $i = 0, 1, \dots$\\
	\textbf{Equality}: $\sum_{i \geq 0}a_iX^i = \sum_{i \geq 0}b_iX^i \iff a_i = b_i$ for all $i \geq 0$\\
	\textbf{Addition}: $\sum_{i \geq 0}a_iX^i + \sum_{i \geq 0}b_iX^i = \sum_{i \geq 0}(a_i + b_i)X^i$\\
	\textbf{Multiplication}: $(\sum_{i \geq 0}a_iX^i)(\sum_{i \geq 0}b_iX^i) = \sum_{k \geq 0}(\sum_{i + j = k}a_ib_j)X^k$\\
	\textbf{Zero element} is $\sum_{i \geq 0}0X^i = 0$ and the \textbf{one} is $1X^0 + \sum_{i \geq 1}0X^i = 1$\\
	
	\noindent
	For a polynomial $f = \sum_{i \geq 0}a_iX^i$, we define the \textbf{degree} of $f$, denoted $deg(f)$, to be the largest $i$ such that $a_i \neq 0$ and we let $deg(f) = -\infty$ if $f = 0$.
\end{framed}

\begin{framed}
	\begin{center}
		\textbf{\textsc{2 Integral Domains and Fields}}
	\end{center}
	The \textbf{characteristic}, $char(R)$, of a ring $R$ is the least positive integer $n$ such that $n \cdot 1 = 0$. If there is no such $n$, then the characteristic of $R$ is defined to be 0.\\
	
	\noindent
	A non-zero element $r \in R$ is a \textbf{zero-divisor} if there is a non-zero element $s \in R$ with $rs = 0$ or $sr = 0$.\\
	
	\noindent
	The ring $R$ is a \textbf{domain} if, for all $r, s \in R$, $rs = 0 \implies r = 0$ or $s = 0$, so a domain is a ring with \textbf{no} zero-divisors. A commutative domain is called an \textbf{integral domain}.\\
	
	\noindent
	A \textbf{division ring} is a ring in which every non-zero element has a right inverse and a left inverse. In this case, these inverses are the same. We write $r^{-1}$ for this \textbf{inverse} of $r$ and say that $r$ is \textbf{invertible} or that $r$ is a \textbf{unit}. A \textbf{field} is a commutative division ring.\\
	
	\noindent
	An element $r$ of a ring $R$ is \textbf{nilpotent} if there is some integer $n \geq 1$ with $r^n = 0$ and the least such $n$ is the \textbf{index of nilpotence} of r. An element $r \in R$ is \textbf{idempotent} if $r^2 = r$ - and 0 and 1 are idempotent in any ring.
\end{framed}

\begin{framed}
	\begin{center}
		\textbf{\textsc{3 Isomorphisms, Homomorphisms and Ideals}}
	\end{center}
	If $R$ and $S$ are rings then an \textbf{isomorphism} from $R$ to $S$ is a \textbf{bijection} $\theta: R \rightarrow S$ such that, for all $r, r' \in R$:\\
	$$\theta(r + r') = \theta(r) + \theta(r') \quad \text{and} \quad \theta(r \times r') = \theta(r) \times \theta(r')$$
	If $\theta$ is an isomorphism from $R$ to $S$, then we write $\theta: R \simeq S$. We say that $R$ and $S$ are \textbf{isomorphic}, and write $R \simeq S$, if there is an isomorphism from $R$ to $S$.\\
	
	\noindent
	If $R$ and $S$ are rings then a \textbf{homomorphism} from $R$ to $S$ is a map $\theta: R \rightarrow S$ such that, for all $r, r' \in R$:
	$$\theta(r + r') = \theta(r) + \theta(r') \quad and \quad \theta(r \times r') = \theta(r) \times \theta(r') \quad and \quad \theta(1_R) = 1_S$$
	
	\noindent
	An \textbf{embedding}, or \textbf{monomorphism}, is an injective homomorphism.\\
	
	\noindent
	If $\theta: R \rightarrow S$ is a homomorphism of rings then the \textbf{kernel} of $\theta$, $ker(\theta)$, is the set $\{r \in R \, \vert \, \theta(r) = 0\}$.\\
	
	\noindent
	An \textbf{automorphism} of a ring is an isomorphism from the ring to itself.\\
	
	\noindent
	An \textbf{ideal} of a ring R is a subset $I \subseteq R$ such that:
	$$0 \in I \quad \text{and} \quad a + b \in I, \; \text{for all} \; a, b \in I \quad \text{and} \quad ar \in I \; \text{and} \; ra \in I \; \text{for all} \; a \in I \; \text{and for all} \; r \in R$$
	
	\noindent
	We write $I \triangleleft R$ to mean that $I$ is an ideal of $R$.\\
	
	\noindent
	If $a \in R$ then $\{r_1as_1 + \cdots + r_nas_n \; \vert \; n \geq 1, r_i, s_i \in R\}$ is an ideal which contains $a$ and is the smallest ideal of $R$ containing $a$. It is called the \textbf{principal ideal generated by} $a$ and is denoted $\langle a \rangle$. If $R$ is commutative, then its description simplifies: $\langle a \rangle = \{ar \; \vert \; r \in R\}$. A \textbf{principal} ideal is one which can be generated by a single element.\\
	
	\noindent
	In every ring $\langle 0 \rangle = \{ 0 \}$ is the smallest ideal and is called the \textbf{trivial ideal}.\\
	In every ring $\langle 1 \rangle = R$ is the largest ideal and every other ideal is referred to as a \textbf{proper ideal}.\\
	
	\noindent
	The more general notion of \textbf{right ideal} is defined as for ideal but with the third condition replaced by the weaker condition: $a \in I$ and $r \in R$ implies $ar \in I$. Then, if $a \in R$, the \textbf{principal right ideal generated by} $a \in R$ is defined to be the set $\{ar \; \vert \; r \in R\}$ and is denoted $aR$.
\end{framed}

\begin{framed}
	\begin{center}
		\textbf{\textsc{4 Factor Rings}}
	\end{center}
	Let $R$ be a ring and let $I$ be a proper ideal. Let $R/I$ denote the set of cosets of $I$ in the additive group $\langle R, + \rangle$, $R/I = \{r + I \; \vert \; r \in R\}$ with operations $+$ and $\times$ defined on $R/I$ as follows:
	$$(r + I) + (s + I) = (r + s) + I \quad \text{and} \quad (r + I) \times (s + I) = (r \times s) + I$$
	This ring is the \textbf{factor ring} (or \textbf{quotient ring}) of $R$ by $I$.\\
	
	\noindent
	\textbf{Fundamental Isomorphism Theorem} Let $I$ be a proper ideal of the ring $R$.\\
	$(i)$ the map $\pi : R \rightarrow R/I$ defined by $\pi(r) = r + I$ is a surjective ring homomorphism with kernel $I$. $\pi$ is called the \textbf{canonical surjection} or \textbf{canonical projection}.\\
	$(ii)$ if $\theta : R \rightarrow S$ is a homomorphism and $I \subseteq ker(\theta)$ then there is a unique map $\theta' : R/I \rightarrow S$ with $\theta' \circ \pi = \theta$. This map $\theta'$ is a homomorphism.\\
	$(iii)$ the map $\theta'$ is injective iff $ker(\theta) = I$. If $\theta$ is surjective and $ker(\theta) = I$ then $\theta'$ is a isomorphism.\\
	
	\noindent
	\textbf{Some other theorem} Let $I$ be an ideal of the ring $R$, then there is a natural, inclusion-preserving, bijection between the set of ideals of $R$ which contain $I$ and the set of ideals of the factor ring $R/I$:\\
	$\bullet$ to an ideal $J \geq I$ there corresponds $\pi J = \{r + I \; \vert \; r \in J\} = \{\pi(r) \; \vert \; r \in J\}$, an ideal in $R/I$\\
	$\bullet$ to an ideal $K \triangleleft R/I$ there corresponds $\pi^{-1}K = \{r \in R \; \vert \; \pi(r) \in K\}$, an ideal in $R$\\
	The notation $J/I$ is also used instead of $\pi J$ for the image of $J$ in $R/I$.\\
	
	\noindent
	An ideal $I$ of a ring $R$ is \textbf{maximal} if it is proper and for any ideal $J$ with $I \leq J \leq R$, then either $J = I$ or $J = R$.\\
	
	\noindent
	\textbf{Another theorem} If $I \leq J$ are ideals of $R$, so $J/I$ is an ideal of $R/I$, then $(R/I)/(J/I) \simeq R/J$.\\
	
	\noindent
	A proper ideal $I$ of a commutative ring $R$ is \textbf{prime} if whenever $r, s \in R$ and $rs \in I$ then either $r \in I$ or $s \in I$.
\end{framed}


\end{document}