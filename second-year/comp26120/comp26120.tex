\documentclass[a4paper]{article}

\usepackage{mathtools}
\DeclarePairedDelimiter\ceil{\lceil}{\rceil}
\DeclarePairedDelimiter\floor{\lfloor}{\rfloor}

\usepackage[a4paper, margin=0.35in]{geometry}
\usepackage{multicol}
\usepackage{amssymb}
\usepackage{amsmath}
\usepackage{centernot}
\usepackage{framed}
\usepackage{bm}
\usepackage{listings}

\begin{document}

\pagenumbering{gobble}
	
\begin{center}
	\huge{\textbf{algorithms and imperative programming (i)}}\\
	\small{Available at \textsc{jtang.dev/resources}}\\
\end{center}
\begin{multicols}{2}

\begin{framed}
	\begin{center}
		\textbf{\textsc{complexity measures}}
	\end{center}
	\textbf{O$(f)$} denotes a set of functions:\\
	$\{g:\mathbb{N} \rightarrow \mathbb{N} \vert \exists n_0 \in \mathbb{N}, c \in \mathbb{R}^+, \forall n > n_0, g(n) \leq c \cdot f(n) \}$
	\textbf{$\bm{\Omega(f)}$} denotes a set of functions:\\
	$\{g:\mathbb{N} \rightarrow \mathbb{N} \vert \exists n_0 \in \mathbb{N}, c \in \mathbb{R}^+, \forall n > n_0, g(n) \geq c \cdot f(n) \}$
	\textbf{$\bm{\Theta(f)}$} denotes $\bm{O(f) \, \cap \, \Omega(f)}$
\end{framed}

\begin{framed}
	\begin{center}
		\textbf{\textsc{euclid's algorithm}}
	\end{center}
\begin{lstlisting}
Algorithm EuclidGCD(a, b):
  Input: Non-negative integers a and b
  Output: gcd(a, b)
		
  if b = 0 then
    return a
  return EuclidGCD(b, a mod b)
end
\end{lstlisting}

\noindent
\textbf{correctness}\\
Let $d = gcd(a, b)$ and $c = gcd(b, a - rb)$. \\
We need to show that $gcd(a, b) = gcd(b, a - rb)$, so $d = c$.\\
By definition of $d$, we have the number $\frac{(a - rb)}{d} = \frac{a}{d} - r\frac{b}{d}$ is an integer as $d \vert a$ and $d \vert b$ and we have also shown $d \vert a - rb$ hence $d \leq c$.\\
Now by definition of $c$, $\frac{a - rb}{c} = \frac{a}{c} - r\frac{b}{c}$ shows that $c \vert a$ as we know $r\frac{b}{c}$ is an integer and $\frac{a - rb}{c}$ is an integer, so we have $c \leq d$.\\

\noindent
\textbf{complexity}\\
After the first call, the first argument is always larger than the second one. Denote $a_i$ as the first argument of the $i$th recursive call of EuclidGCD. It is clear that the second argument of a recursive call is equal to $a_{i + 1}$ and we also have $$a_{i+2} = a_i \, \text{mod} \, a_{i+1}$$ which implies the sequence $a_i$ is strictly decreasing. We claim that $$a_{i+2} < \frac{1}{2}a_i$$

\noindent
\textbf{case 1}: $a_{i+1} \leq \frac{1}{2}a_i$, since the sequence of $a_i$'s is strictly decreasing, we have $$a_{i+2} < a_{i+1} \leq \frac{1}{2} a_i$$

\noindent
\textbf{case 2}: $a_{i+1} > \frac{1}{2}a_i$, in this case $a_{i + 2} = a_i \, \text{mod} \, a_{i+1}$, so we have $$a_{i+2} = a_i \, \text{mod} \, a_{i+1} = a_i - a_{i + 1} < \frac{1}{2}a_i$$

\noindent
Thus the size of the first argument to the EuclidGCD method decreases by half with every other recursive call. Hence we have $\bm{O(log \max(a, b))}$.
\end{framed}

\begin{framed}
	\begin{center}
		\textbf{\textsc{modular arithmetic}}
	\end{center}
\begin{lstlisting}
Algorithm pow1(a, b, k):
  Input: Integers a, b, k
  Output: a^b mod k
		
  s = 1
  for i from 1 to b
    s = s * a mod k
  return s
end
\end{lstlisting}
\noindent
The number of operations performed here is clearly $O(b)$, therefore the time complexity is $O(2^n)$ as the size of $b$ is $\log_2 b$.\\
\begin{lstlisting}
Algorithm pow2(a, b, k):
  Input: Integers a, b, k
  Output: a^b mod k
		
  d = a, e = b, s = 1
  until e = 0
    if e is odd
      s = s * d mod k
    d = d * d mod k
    e = floor(e / 2)
  return s
end
\end{lstlisting}
\noindent
The number of operations performed here is proportional to the number of times $e$ ($=b$) can be halved before reaching 0, i.e. at most $\ceil{log_2b}$. It follows that this algorithm has running time in $O(n)$.\\

\noindent
\textbf{primitive roots}\\
We say that $g$ is a \textbf{primitive root} with respect to $p$ means that $\mathbb{Z}_p = \{1, 2, \cdots, p - 1\}= \langle g \rangle = \{g^i \, \text{mod} \, p \, \vert \, i \in \mathbb{Z}\}$\\

\begin{lstlisting}
Algorithm dl(y, g, p):
  Input: Integers y, g, p
  Output: x such that y = g^x mod p
	
  a = y mod p	
  for x from 1 to p - 1
    b = pow2(g, x, p)
    if a = b
      return x
  end
end
\end{lstlisting}

\noindent
The number of loop iterations is $O(p)$ and in each iteration, the pow2 call is $O(x)$. So the total number of operations is bounded by $O(px)$ but $x < p$ so this is also bounded by $O(p^2)$ which is $O(4^n)$ as the size of $p$ is $log_2p$.\\

\noindent
\textbf{El Gamal} with private key $x$\\ 
public key $(p, g, y)$ with $y = g^x \mod p$\\
cipher $(a, b)$ with $a = g^k \mod p$ and $b = My^k \mod p$\\
message $M$ = $b/(a^x) \mod p = b(a^x)^{-1} \mod p$
\end{framed}

\newpage

\begin{framed}
\begin{center}
	\textbf{\textsc{bubble sort}}
\end{center}
\begin{lstlisting}
Algorithm bubbleSort(A):
  Input: An (unsorted) array A
  Output: An sorted array A
  
  n = length(A)
  swapped = true
  while swapped
    swapped = false
    for i from 0 to n
      if a[i] > a[i + 1]
      	swap(a[i], a[i + 1])
      	swapped = true
end
\end{lstlisting}

\noindent
For each element in the array, bubbleSort does $n - 1$ comparisons which is $O(n)$ and there are $n$ elements in the array so bubbleSort has a total running time of $\bm{O(n^2)}$.
\end{framed}

\begin{framed}
\begin{center}
	\textbf{\textsc{merge sort}}
\end{center}
\begin{lstlisting}
Algorithm merge(L, R):
  Input: Two sorted arrays L and R
  Output: An sorted array of L and R
  
  if L = []
    return R
  if R = []
    return L
  a = L[1], b = R[1]
  L' = L without a, R' = R without b
  if a <= b
    return [a] + merge(L', R)
  return [b] + merge(L, R')
end
\end{lstlisting}

\noindent
When merge(L, R) is called, at most one recursive call is made, in which $\vert L \vert + \vert R \vert$ decreases by 1. Therefore, at most $O(n)$ recursive calls are made, where $n = \vert L \vert + \vert R \vert $ is the length of the input and since a constant number of operations are executed for each recursive call, it takes at most $O(n)$ time to run.\\

\begin{lstlisting}
Algorithm mergeSort(X):
  Input: An (unsorted) array X
  Output: An sorted array X
  
  if |X| <= 1
    return X
  split X into two halves, X = L + R
  return merge(mergeSort(L), mergeSort(R))
end
\end{lstlisting}

\noindent
The total lengths of lists processed at each level of recursion is constant at $\vert X \vert = n$ and the total amount of work done for each call is linear in the lengths of the arguments. The number of times $X$ can be halved is $O(\log n)$ hence the time complexity of mergeSort is $\bm{O(n \log n)}$.
\end{framed}

\begin{framed}
	\begin{center}
		\textbf{\textsc{quick sort}}
	\end{center}
\noindent
In the algorithm, p will be our pivot.
	\begin{lstlisting}
Algorithm quickSort(L):
  Input: Array to be sorted L
  Output: An sorted array of L
	
  if length(L) <= 1
    return L
  remove first element, p, from L
  A = elements in L that are <= p
  B = elements in L that are > p
  L = quickSort(A)
  R = quickSort(B)
  return L + p + R
end
\end{lstlisting}
	
\noindent
The worst case occurs when for each recursive call, one of A or B is empty.
Let $n$ be the size of our array L.
Then $n$ recursive calls are made, with the argument one element shorter each time.
Before each recursive call, A and B must be calculated which requires $O(n)$ steps.
So the total work done is $n+(n-1)+...+1=\frac{1}{2}n(n+1)$.
Hence quick sort is in $\bm{O(n^2)}$.

\end{framed}

\begin{framed}
	\begin{center}
		\textbf{\textsc{bucket sort}}
	\end{center}
\noindent
Suppose we wanted to sort $n$ items whose keys are integers in the range $[0, N - 1]$ for some integer $N \geq 2$. For example, we want to sort the two-digit numbers $[15, 45, 10, 30, 25, 28, 15, 50, 36]$ into ascending order of the first digit then bucket sort will return $[15, 10, 15, 25, 28, 30 , 36, 45, 50]$. Some implementations will use another algorithm to sort each bucket itself. 

\begin{lstlisting}
Algorithm bucketSort(S):
  Input: S with keys in [0, N - 1]
  Output: S sorted in order of keys

  B array of N empty lists
  foreach x in S
    k = key of x
    remove x from S
    add x to B[k]
  for i = 1 to N
    sort(B[i])
  for i = 1 to N
    for each x in B[i]
      remove x from B[i]
      add x to end of S
end
\end{lstlisting}
	
\noindent
The worse case for bucket sort is when all elements are allocated to the same bucket and we get $\bm{O(n^2)}$. Since individual buckets are sorted using another algorithm, if only a single bucket needs to be sorted, bucket sort will take on the complexity of the inner sorting algorithm.

\end{framed}

\newpage

\begin{framed}
	\begin{center}
		\textbf{\textsc{determinants and permanents}}
	\end{center}
	\textbf{permutations}\\
	A \textbf{permutation} is a 1-1 map of a set X onto itself. The number of permutations on an $n$-element set is $n!$. A simple inductive proof shows that, for all $n \geq 4$, $2^n \leq n! \leq 2^{n^2}$. That is: $n \mapsto n!$ is $\Omega(2^n)$ and $O(2^{n^2})$.\\
	
	\noindent
	\textbf{transpositions}\\
	A \textbf{transposition} is a permutation of two elements, i.e $\sigma = (\alpha\beta)$.
	
	\noindent
	\textbf{parity}\\
	The \textbf{parity} of a permutation $\sigma$, denoted $sgn(\sigma)$ is 1 if $\sigma$ is the product of an even number of transpositions, -1 otherwise.\\
	
	\noindent
	\textbf{proof that $sgn(\sigma) = \pm 1$}\\
	Let $\sigma \in S_n$. Write $\sigma = (\alpha_1 \alpha_2 \cdots \alpha_m)$. In the view of
	$$(\beta_1 \beta_2 \beta_3 \cdots \beta_k) = (\beta_1 \beta_2)(\beta_1 \beta_3) \cdots (\beta_1 \beta_k)$$
	any permutation of length $k$ can be written as a composite of $k - 1$ transpositions. Now consider when $k$ is even or odd, then the result follows.\\
	
	\noindent
	Note that $sgn(\sigma \cdot \tau) = sgn(\sigma) \cdot sgn(\tau)$ and $sgn(t) = -1$, where $t$ is a transposition.\\
	
	\noindent
	\textbf{matrices}\\
	$$A = 
 	\begin{pmatrix}
 	 a_{1,1} & a_{1,2} & \cdots & a_{1,n} \\
 	 a_{2,1} & a_{2,2} & \cdots & a_{2,n} \\
 	 \vdots  & \vdots  & \ddots & \vdots  \\
 	 a_{n,1} & a_{n,2} & \cdots & a_{n,n} 
 	\end{pmatrix}$$
 	
 	$$det(A) = \sum_{\sigma \in \text{perm}\{1, \cdots, n\}} sgn(\sigma)\prod^n_{i=1}a_{i,\sigma(i)}$$
 	$$permanent(A) = \sum_{\sigma \in \text{perm}\{1, \cdots, n\}} \prod^n_{i=1}a_{i,\sigma(i)}$$
	
	\noindent
	\textbf{calculating the determinant}\\
	We can convert the matrix into upper triangular form, then we know the determinant is the product of the elements across the diagonal. To zero the $i$ column below the diagonal, we need to do a transposition of columns $O(n)$ and for each of $(n - i + 1) \leq n$ rows below the $i$th row, subtract a multiple of the $i$th row from that row, which contains $n - i + 1 \leq n$ non-zero elements, so $O(n)$ operations per row. So cost of zeroing the $i$th column below the diagonal is $O(n^2)$. There are $n - 1 \leq n$ columns to zero below the diagonal, so cost of converting to UT form is $O(n^3)$ and cost of multiplying diagonal elements is $O(n)$. Hence total cost is $O(n^3 + n) = \bm{O(n^3)}$.\\
	
	\noindent
	\textbf{calculating the permanent}\\
	There are no efficient methods to calculate the permanent - the best-known algorithms run in exponential time.
\end{framed}

\begin{framed}
\begin{center}
	\textbf{\textsc{lexicographic order}}
\end{center}

\noindent
Consider a finite set A which is totally ordered. Given two different elements of the same length $\alpha_1\alpha_2\cdots\alpha_k$ and $\beta_1\beta_2\cdots\beta_k$, the first sequence is smaller than the second one for lexicographic order, if $a_i < b_i$ for the first $i$ where $a_i$ and $b_i$ are different.\\

\noindent
If one sequence is shorter than another, then pad it with "blank" characters - a character than is treated as smaller than every element of $A$.
\end{framed}

\begin{framed}
\begin{center}
	\textbf{\textsc{$\bm{\Omega(n\log n)}$ for comparison-based algorithm}}
\end{center}

\noindent
Suppose we want to sort $n$ elements. There are $n!$ permutations of these $n$ elements. If we draw a binary tree with each leaf represents a permutation of these $n$ elements, the number of comparisons we need at most is the height of tree. A tree with height $h$ has at most $2^h$ leaves, then we have $n! \leq 2^h$, it follows that $log(n!) \leq h$. In the view of $$n! > (\frac{n}{2})^\frac{n}{2} \, \text{for} \, n \geq 1$$ 

\noindent
we know that $h \geq log(n!) \geq log (\frac{n}{2})^\frac{n}{2} = (\frac{n}{2})\, log \, \frac{n}{2}$ so it follows that $h \in \Omega(n \log n)$.
\end{framed}

\begin{framed}
\begin{center}
	\textbf{\textsc{notes}}
\end{center}
\vspace{109mm}
\end{framed}
\end{multicols}
\end{document}