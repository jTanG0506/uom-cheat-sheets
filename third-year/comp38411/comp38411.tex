\documentclass[a4paper]{article}

\usepackage[a4paper, margin=0.35in]{geometry}
\usepackage{multicol}
\usepackage{amssymb}
\usepackage{amsmath}
\usepackage{mathtools}
\usepackage{centernot}
\usepackage{framed}

\newcommand\abs[1]{\ensuremath{\lvert#1\rvert}}
\newcommand\divides{\ensuremath{\big|}}

\begin{document}

\pagenumbering{gobble}
	
\begin{center}
	\huge{\textbf{comp38411 - cryptography and network security}}\\
	\small{available at \textsc{jtang.dev/resources}}\\
\end{center}

\begin{multicols}{2}

\begin{framed}
	\begin{center}
		\textbf{\textsc{(1) introduction}}
	\end{center}
	
	\noindent
	\textbf{confidentiality}: preventing disclosure of information\\
	\textbf{integrity}: preventing unauthorised modifications\\
	\textbf{availability}: ensuring timely and reliable access\\
	\textbf{authenticity}: being genuine and able to be verified\\
	\textbf{accountability}: actions of an entity can be traced\\
	
	\noindent
	\textbf{masquerade}: impersonation of another entity\\
	\textbf{replay}: subsequent transmission of captured data unit\\
	\textbf{modification of messages}: alteration of messages\\
	\textbf{denial of service}: prevent normal use of facilities
\end{framed}

\begin{framed}
	\begin{center}
		\textbf{\textsc{(2) introduction to cryptography}}
	\end{center}
	
	\noindent
	\textbf{symmetric encryption}: $C = E(K, P), \;P = D(K, C)$\\
	
	\noindent
	\textbf{confusion}: each bit of the ciphertext has a non-linear relationship with each bit of the plaintext bits and key bits\\
	\textbf{diffusion}: changes to plaintext bits or key bits spreads and affects many ciphertext bits\\
	
	\noindent
	\textbf{stream ciphers}: encrypt one bit or one byte at a time, uses XOR operator\\
	
	\noindent
	\textbf{unconditionally secure}: cannot be broken no matter what. only one-time pad is unconditionally secure - but is highly impractical\\
	\textbf{provably secure}: the difficulty of breaking the system is at least as difficult as a known hard problem, such as integer factorisation. examples would be RSA and DSA\\
	\textbf{ad-hoc security}: not provably secure, but we trust it as many skilled people have failed to break it, such as AES\\
	
	\noindent
	\textbf{ciphertext-only attack}: attacker only has ciphertext itself to analyse, hard to do much due to lack of info\\
	\textbf{known-plaintext attack}: attacker has plaintext-ciphertext pairs at their disposable\\
	\textbf{chosen-plaintext attack}: attacker is able to choose plaintext to encrypt and deliberately picks patterns to reveal structure
\end{framed}

\begin{framed}
	\begin{center}
		\textbf{\textsc{(3) - conventional cryptography}}
	\end{center}
	
	\noindent
	\textbf{security goals of encryption algorithm}\\
	(1) completeness - every bit of the output should depend on every bit of the input and every bit of the key\\
	(2) avalanche effect - small change in either plaintext or the text produces significant change in the ciphertext\\
	(3) statistical independence - the ciphertext and plaintext appear to be statistically independent (no patterns\\
	
	\noindent
	\textbf{block ciphers}: work with blocks of data at a time, defined by the block size. the larger the block size, the larger the memory footprint, and so less ideal for embedded hardware, say.
\end{framed}

\begin{framed}
	\begin{center}
		\textbf{\textsc{(3) - conventional cryptography}}
	\end{center}
	
	\noindent
	\textbf{DES}: a \textbf{16 round} feistel cipher, which takes a \textbf{64 bit input}, and a \textbf{56 bit key}. operations include permutation, XOR and substitution\\
	
	\noindent
	\textbf{AES}: a \textbf{10/12/14 round} cipher, which takes a \textbf{128 bit input} with \textbf{128/192/256 bit key}. operations include AddRoundKey, SubBytes, ShiftRows, MixColumns.\\
	
	\noindent
	\textbf{why is AES the way it is?}\\
	AddRoundKey: ensures encryption depends on key\\
	SubBytes: provides non-linear relationship (confusion)\\
	ShiftRows: changes in a given column affect other columns (diffusion / avalanche)\\
	MixColumns: changes in a byte affect other bytes in the state (diffusion)\\
	KeyExpansion: otherwise the same key is used in every round, thus being vulnerable to a slide attack\\
	
	\noindent
	\textbf{ECB mode}: plaintext blocks are processed independently, \textit{extremely insecure}. no error propagation onto subsequent blocks as blocks are processed independently.\\
	
	\noindent
	\textbf{CBC mode}: chaining of blocks by the means of feedback, which aids in concealing the patterns of plaintext. errors in one block will propagate to subsequent blocks. encryption is not parallelisable, but decryption is.\\
	
	\noindent
	\textbf{CTR mode}: encrypts a nonce along with an incrementing counter value, which serves as a key for XOR'ing with the plaintext block. can be used to turn a block cipher into a stream cipher.
\end{framed}

\begin{framed}
	\begin{center}
		\textbf{\textsc{(4) public key cryptography}}
	\end{center}
	
	\noindent
	\textbf{asymmetric encryption}: makes use of a pair of keys (public and private). $C = E(K_E, P)$ and $P = E(K_D, C)$. asymmetric encryption is more expensive than symmetric!\\
	
	\noindent
	\textbf{rsa}: can be used for encryption and signatures\\
	\textbf{key generation}\\
	(1) choose two large primes $p, q, p \neq q$\\
	(2) calculate $n = pq$ and $\varphi(n) = (p - 1)(q - 1)$\\
	(3) choose $e \in \mathbb{Z}, gcd(\varphi(n), e) = 1, e \in (1, \varphi(n))$\\
	(4) calculate $d \equiv e^{-1} \mod \varphi(n)$\\
	private key = \{d, n\} and public key = \{e, n\}\\
	\textbf{encryption} is $C \equiv M^e \mod n$ ($M \in [0, n -1]$)\\
	\textbf{decryption} is $M \equiv C^d \mod n$
\end{framed}

\begin{framed}
	\begin{center}
		\textbf{\textsc{(5) mac and hash functions}}
	\end{center}
	
	\noindent
	\textbf{hash function}: takes an input of any length and produces a fixed size output, known as the hash value.\\
\end{framed}

\begin{framed}
	\begin{center}
		\textbf{\textsc{(5) mac and hash functions}}
	\end{center}

	\noindent
	\textbf{four properties of a good hash function}\\
	(1) compression: takes an input of any size and produced a fixed length output\\
	(2) one-way: easy to compute $h = H(x)$, but impossible to compute $x$ from $H(x)$\\
	(3) second preimage: given $m_1$, it is impossible to find $m_2 \neq m_1$ wit h $H(m_1) = H(m_2)$\\
	(4) strong collision: it is impossible to find $m_1 \neq m_2$ with $H(m_1) = H(m_2)$
	
	\noindent
	if $H$ is not second preimage resistant, signature forgery is possible. if $H$ is not strong collision resistant, we don't have non-repudiation\\
	
	\noindent
	\textbf{MAC}: a function which protects the integrity and authenticity of a message, by producing a tag\\
	\textbf{fresh message}: not a replayed message, can be achieved by sequence numbers, timestamps and nonces\\
	\textbf{keyed hash}: hashing the message and key, $h = H(K \vert \vert M)$\\
	\textbf{HMAC}: MAC construction based on hash function\\
	
	\noindent
	\textbf{types of authenticated encryption}: hash-then-encryption, MAC-then-encryption, encrypt-then-MAC, encrypt-and-MAC. only encrypt-then-MAC allows the recipient to verify the tag before decrypting - anti-DoS.\\
	
	\noindent
	\textbf{CMAC}: MAC based on CBC mode, take the last output block as the tag (technically, the MSBs of this last block)
\end{framed}

\begin{framed}
	\begin{center}
		\textbf{\textsc{(6) digital signatures}}
	\end{center}
	
	\noindent
	\textbf{digital signatures} should be message-dependent (non-reusable / integrity), signer-independent (unforgeable / authenticity) and verifiable (origin authenticity)\\
	
	\noindent
	\textbf{DSA}: signature algorithm based on discrete logarithms. does not provide encryption like RSA. message digest cannot be recovered from signature, unlike RSA. requires an ephemeral key for each message.
\end{framed}

\begin{framed}
	\begin{center}
		\textbf{\textsc{(7) public key infrastructure}}
	\end{center}
	
	\noindent
	\textbf{PKI}: set of protocols, policies and procedures to enable practical deployment of public key cryptography\\
	
	\noindent
	\textbf{end entity}: end-users, devices or any identifiable entity\\
	\textbf{certificate authority}: issuer of certificate and certificate revocation lists, should be a trustworthy source
	
	\noindent
	\textbf{mandatory fields of a x.509 certificate}\\
	version number, serial number, signature algorithm ID, issuer, validity period, subject, subject public key, signature\\
	
	\noindent
	\textbf{certificate revocation list}: list of blacklisted certificates before its expiry date. reasons for revocation:\\
	(1) user's private key compromised\\
	(2) user is no longer certified by the CA\\
	(3) CA's certificate has been compromised
\end{framed}

\begin{framed}
	\begin{center}
		\textbf{\textsc{(7) public key infrastructure}}
	\end{center}
	
	\noindent
	\textbf{verifying a certificate}: follow the issuance chain in a bottom-up manner, until you find a certificate with a signature you trust. number of levels required to check can be reduced by means of cross certification.\\
	
	\noindent
	\textbf{digital certificate}: an document which binds the public key of an entity to the entity itself
\end{framed}

\begin{framed}
	\begin{center}
		\textbf{\textsc{(8) key management}}
	\end{center}
	
	\noindent
	\textbf{key management issues}\\
	(1) generation: are the generated keys of sufficient length\\
	(2) storage: are keys stored in a secure manner\\
	(3) transportation: are keys delivered in a securely\\
	(4) establishment: how do two parties establish keys\\
	(5) revocation: how are keys revoked and replaced?\\
	
	\noindent
	\textbf{key generation}: the key space should be sufficiently large, else a exhaustive search is feasible. keys should be generated using psuedo-random generators, not some lame-ass random number generator.\\
	
	\noindent
	\textbf{key storage}: keys need to be stored securely. key wrapping is encrypting the key with another key - keys should never appear unencrypted outside the encryption device. hardware tokens should require a PIN to access from secure memory.\\
	
	\noindent
	\textbf{session keys}: use session keys so that adversaries have less ciphertext for cryptanalysis. shorter lifetime of session key limits the exposure in the event of key compromise.\\
	
	\noindent
	\textbf{anonymous diffie-hellman}: public-key exchange protocol. both parties know $g$ and $p$ (prime). each party chooses a random exponent, say $a$ and $b$, computes $g^a$ and $g^b$ respectively, and sends it to the other party. upon receipt, the shared key can calculate by $(g^a)^b$ and $(g^b)^a$. this is vulnerable to \textbf{man-in-the-middle} attack.\\
	
	\noindent
	\textbf{authenticated diffie-hellman}: the communicating parties sign the $g^a$ and $g^b$ values, so that a man in the middle cannot modify these values whilst retaining a valid signature\\
	
	\noindent
	\textbf{key distribution without PKC}: use a trusted third party (KDC) which is assumed to have shared master secrets with the two parties which wish to communicate. an example would be needham-schroeder protocol.\\
	
	\noindent
	\textbf{nonces vs timestamps}: nonces need to be (at least partially) contributed from the recipient - so we cannot do this if say, the recipients is a large and/or unknown group. nonces should be psuedo-random and from a large key space, else we will have nonce reuse. timestamps require communicating principles to have their clocks in sync. both are used to achieve freshness.
\end{framed}

\newpage

\begin{framed}
	\begin{center}
		\textbf{\textsc{(9) entity authentication}}
	\end{center}
	
	\noindent
	\textbf{entity authentication} is the process whereby one party is assured of the identity of a second party.\\
	
	\noindent
	\textbf{basis of identification}\\
	(1) something known (passwords, pins)\\
	(2) something possessed (chip cards) not demons!\\
	(3) something inherent (fingerprints, retinal patterns)\\
	
	\noindent
	\textbf{AAA services}\\
	\textbf{authorisation}: linking access rights to particular identities\\
	\textbf{authentication}: verifying the identity of a claimant\\
	\textbf{accounting}: using a users identity for logging security-related events\\
	
	\noindent
	\textbf{how unix authentication works}\\
	(1) user signs up with a password, unix computes a salt based on the current time, hashes the password and the timestamp, to produce a hashed password. the timestamp is also stored\\
	(2) when the user tries to login, unix will hash the user input, along with the timestamp, and compare it against the hashed password stored\\
	
	\noindent
	\textbf{how unix protects users passwords}\\
	(1) hashing: plaintext passwords are not stored\\
	(2) password salting: the salt alters the DES expansion function, meaning each password in a dictionary attack requires 4096 trials. in addition, off-the-shelf DES chips cannot be used to assist hardware based attacks\\
	(3) shadow files: the hashed password is stored in a shadow file which only is accessible with root privileges\\
	
	\noindent
	\textbf{replay attacks on unix}\\
	unix is not protected against replay attacks, attackers can eavesdrop on a network and later replay it to gain access. we assume LAN is secure and that secures do not bring their own client software in, and often overlook this issue\\
	
	\noindent
	\textbf{one-time passwords}: safe from eavesdroppers and subsequent impersonation. can be done using lamport's OTP scheme, hashing-based OTP, etc
\end{framed}

\begin{framed}
	\begin{center}
		\textbf{\textsc{(10) ip security}}
	\end{center}
	
	\noindent
	\textbf{IPSec} provides security at the IP level, by the means of \textbf{authentication header (AH)} or \textbf{encapsulating security payload (ESP)}\\
	
	\noindent
	\textbf{AH} provides access control, connectionless integrity, data origin authentication and rejection of replayed packets\\
	\textbf{ESP} provides access control, confidentiality, limited traffic confidentiality\\
	\textbf{ESP with authentication} provides everything AH and ESP provide\\
	
	\noindent
	\textbf{security association (SA)}: a set of agreed attributes (authentication mechanism, encryption algorithm, etc) between two parties that wish to communicate\\
	
	\noindent
	\textbf{transport mode}: regular mode for packets to travel from source to destination in network and \textbf{only applicable to host implementations}\\
	
	\noindent
	\textbf{tunnel mode}: can be used if the endpoints do not have the ability to carry out security checks on the packets, then the IPSec Server is responsible for doing the checks\\
	
	\noindent
	\textbf{anti-replay}: achieved using sequence numbers, attached by the sender. upon receipt (if they ever arrive), we use a \textit{window} to help reorder the packets\\
	
	\noindent
	\textbf{combining SA's}\\
	as a single SA can only implement either AH or ESP \textbf{but not both}, sometimes we employ multiple SA's to achieve certain IPSec services. for example, transport-tunnel bundle is when you apply AH transport first, then ESP tunnel. benefit of transport-tunnel over ESP with authentication is that more fields are protected, namely the original IP header
\end{framed}

\begin{framed}
	\begin{center}
		\textbf{\textsc{full notes}}
	\end{center}
	
	\noindent
	detailed notes: \verb|https://comp38411.jtang.dev|
\end{framed}

\end{multicols}

\end{document}